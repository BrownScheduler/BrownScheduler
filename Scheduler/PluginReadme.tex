\documentclass[11pt]{article}

\usepackage{fullpage}

\parskip 7pt
\parindent 0pt

\title{Final Part 2}
\author{Matt Mahoney \\
	Patrick Clay \\
	Aswin Karumbunathan}
\date{10 May 2011}

\begin{document}

\maketitle

\title{So you want to write a plugin}

The TMNT Scheduler was designed explicitly to give people who wanted to create their own tournament management
systems that power.

Note: this guide expects that you be fairly fluent in basic Java concepts such as Interfaces, Classes, Objects
and Attributes. The most difficult concept that will be necessary in order ot write a plugin is that of Reflections,
but don't worry if you don't quite get it yet; we'll walk you through.

\section{Units}

The basic Object that gives plugins their power are the Unit objects described in the Interface. These Units
can be used as replacements for people(e.g. creating Teams, or Judges), or as a marker (e.g. marking a Winner).
If you're creative, you'll find lots more uses for the Unit interface!

Let's start writing a basic plugin. The first thing I know is that I want a Tournament comprised of different people.
Therefore, I'll write a new Unit!

\lstset{language=Java}

\section{Attributes}

\section{Groups}

\subsection{Pairings}

\section{Rounds}

\section{Tournament!}



\end{document}
